\chapter{A framework to improve Ceccato et al.’s work.}
\label{proposal}

\linenumbers
\lettrine[lines=2]{I}{n Chapter \ref{state_of_art}}, I presented the state of the art of \textit{process comprehension} of an Industrial Control System (ICS) focusing later on the methodology proposed by Ceccato et al. \cite{ceccato}[Section \ref{sec:ceccato_metodology}], explaining what it consists of, its practical application on a testbed, and most importantly highlighting its limitations and critical issues (see Section \ref{subsec:ceccato_limitations}).

\bigskip
In this chapter I will present \textbf{my proposals to improve the methodology} presented in the previous chapter, overcoming (or at least trying to do so) the criticalities mentioned above by almost completely rewriting the original framework, enhancing its functionalities and inserting new ones where possible, while keeping its general structure and approach: the system analysis will in fact consist of the same four steps as in the original methodology (Data Pre-processing, Graph and Statistical Analysis, Business Process Mining and Invariants Inference), but each of them will be deeply revised in order to provide a richer, clearer and more complete process comprehension of the industrial system to be analyzed and its behavior.

\bigskip
As it may have already been noted, my proposals do not involve improving the data gathering phase: this is due simply to the fact that the novel framework will not be tested on the same case study used by Ceccato et, al. (Section \ref{subsec:ceccato_testbed}), but on a different case study, the ITrust SWaT system \cite{swat_home}, of which (some) datasets containing the execution trace of the physical system and the network traffic scan are already provided by iTrust itself. For more details about this case study, see Chapter \ref{casestudy}.

\section{Phase 0: General Improvements}
\label{sec:improve_general}

The implementation of the novel framework for ICSs analysis starts from several assumptions:

\begin{enumerate}
	\item it must be implemented in a \textbf{single programming language}
	\item it must be \textbf{independent of the system} to be analyzed
	\item It must provide greater \textbf{flexibility and ease of use} for the user at every stage
\end{enumerate}
In the following subsections, these three points will be discussed in more detail.

\subsection{Single Programming Language}
\label{subsec:framewirf_language}
The original tool was implemented using various programming languages in each of the different phases: from Python up to Java, passing through Bash scripting. \newline
In my opinion, this heterogeneity makes it more difficult and less intuitive for the user to operate on the tool: moreover, the use of multiple technologies makes it more difficult to maintain the code and add new features, particularly if only a single person is managing the code (he/she might be proficient in one language, but little of the others).

\bigskip
For these reasons, I decided to use a single programming language, to ensure homogeneity to the framework and ease of use and maintenance of the code for anyone who wants to manage it in the future: I chose to use Python, because of its simplicity and easy readability combined with its versatility and powerfulness: moreover, Python can count on a massive number of available libraries and packages that meet all kinds of needs.

\vfill
\subsection{System Independence}
\label{subsec:framework_independence}
One of the biggest limitations of Ceccato et al.'s tool that I highlighted in Section \ref{subsec:ceccato_limitations} is the fact that it is \textbf{highly dependent on the testbed used}: that is, it is \textit{not} possible to configure any of the tool's parameters to analyze different industrial systems.\newline
To overcome this issue and make my framework independent of the system to be analyzed, also eliminating all references to hardcoded variables and values present in the previous tool, I decided to use a \textbf{general configuration file}, named \textit{config.ini}, in which the user can, at will, customize all the parameters necessary to perform the analysis of the targeted system.

\paragraph{General Configuration File \textit{config.ini}} 
The configuration file \textit{config.ini} for the framework I am presenting in this thesis is intended, as mentioned, to make it customizable in order to fit a variety of different systems and allow for their analysis. Here the user can configure general parameters and options, such as paths to read from or write files to, or related to individual analysis phases.\newline
The file is divided into sections, each covering a different aspect of the configuration: each section contains user-customizable constants that will then be called within the Python scripts that constitute the framework. An example can be appreciated in Listing \ref{lst:config_ini_example}:

\begin{lstlisting}[language=bash, numbers=none, caption=Example of \textit{config.ini} file, label=lst:config_ini_example]	
	[PATHS]
	root_dir = /home/marcuzzo/UniVr/Tesi
	project_dir = %(root_dir)s/PLC-RE
	input_dataset_directory = %(root_dir)s/datasets_SWaT/2015
	net_csv_path = %(root_dir)s/datasets_SWaT/2015/Network_CSV
	
	[DEFAULTS]
	dataset_file = PLC_SWaT_Dataset.csv
	granularity = 10
	number_of_rows = 20000
	skip_rows = 100000
	
	[DAIKON]
	daikon_dir = daikon
	daikon_invariants_dir = %(daikon_dir)s/Daikon_Invariants
	daikon_results_dir = %(daikon_invariants_dir)s/results
	daikon_results_file_original = daikon_results_full.txt
	inv_conditions_file = Inv_conditions.spinfo
	max_security_pct_margin = 1
	min_security_pct_margin = 2
	
	...
\end{lstlisting}

\subsection{Flexibility and Ease on Use}
\label{subsec:framework_flex}
The lack of flexibility and ease of use in a tool can be a significant disadvantage, limiting its effectiveness and making it challenging for the user to get the desired outcomes. The original tool suffered from these limitations, with users having to run scripts from the command line, with little to no options or parameters available to customize the analysis. As a result, the tool was not user-friendly and lacked the flexibility to adapt to specific user needs. 

\bigskip
To settle these issues, I enhanced the command-line interface in the novel framework by adding new options and parameters. These new features provide the user with greater flexibility, enabling to specify parameters and options that allow for more in-depth analysis and focused results analyzing data more effectively and efficiently. With these enhancements, the framework has become more user-friendly, reducing the learning curve and making it more accessible to a wider range of users. \newline
This, in turn, makes the framework more valuable and useful, increasing its adoption and effectiveness across a range of industries and applications. \newline
Moreover, with new options and parameters users no longer have to rely solely on the command line interface, which can be challenging and intimidating for those with limited technical expertise. Instead, users can now access a range of customizable options and parameters, making the tool more intuitive and user-friendly. 

\bigskip
Overall, the enhancements made to the framework represent a significant step forward in making it more effective, efficient, and user-friendly.

\section{Phase 1: Data Pre-processing}
\label{sec:improve_preprocessing}
\textit{Data Pre-processing phase} is probably the most delicate and significant one: depending on how large the industrial system to be analyzed is, the data collected, and how it is enriched using the additional attributes, the subsequent system analysis will provide more or less accurate outcomes.

\bigskip
The previous tool has many limitations, especially at this stage: it is not possible to isolate a subsystem (either on a temporal basis or on the number of PLCs to be analyzed - the system is considered in its whole), and many of the additional attributes were actually added manually: moreover, for those automatically entered, there is no way to specify which register type to associate the additional attribute with.\newline
All this, combined with the fact that in the tool code many references to attributes and registers are hardcoded, makes the analysis of the system much more difficult and the obtained results less accurate in terms of quantity and quality.

\bigskip
In the novel framework these problems have been overcome by introducing the possibility, starting from the datasets of individual PLCs obtained from data gathering process, to select a subsystem from the command line both on a temporal basis and of the PLCs to be considered; I have also redesigned the whole process of enrichment of the resulting dataset, eliminating the manual entry of additional attributes and giving the user the possibility to be able to decide which type of additional attribute to associate with a given register.\newline
In addition to this, at the end of the pre-processing operation, it is possible to perform a brief preliminary analysis of the obtained dataset in order to estimate which registers are connected to actuators, which to measurements, and which represent hardcoded relative setpoints or constants: this operation also makes it possible to be able to refine the enrichment step by setting the relevant parameters in the \textit{config.ini} file\newline \newline 
In the next sections we will look in more detail at what has been accomplished.

\subsection{Subsystem Selection}
\label{subsec:4_select_subsystem}

In the previous tool, the datasets in CSV format referring to each single PLC are placed in a fixed directory (hardcoded in the script) from which the dedicated script later perform merge and enrichment of them all, resulting in a single dataset representing the entire process trace of the industrial system as an output. As mentioned, the script makes no provision to choose the individual PLCs to be analyzed, nor to decide on a temporal range over which to perform the analysis: in fact, it may happen that during the period of scanning and data gathering there is a so-called \textit{transient}, i.e., a general state in which the industrial system is still initializing before actually reaching full operation; or, more simply, there is the need to analyze the process of only a specific part of the industrial system in a certain period of interest: whatever the motivation, the lack of elasticity and options to provide to the user makes the analysis much more complex than it might be, affecting even the later phases, as the number of variables to be analyzed becomes enormously higher.\newline
In addition, it is not possible to specify an output CSV file where to save the resulting dataset: at each dataset creation and enrichment operation, therefore, the resultant file will be overwritten. This is very awkward when making comparisons between two different execution traces, for example, unless the files are renamed manually.

\bigskip
Let's see how all these issues were solved in the novel framework I developed: first of all, in the general \textit{config.ini} file there are some general default settings about paths, and among them the one concerning the directory where to place the datasets of the individual PLCs to be processed. In addition to this option, there are other ones that define further aspects related to the operations performed in this phase. Listing \ref{lst:config_ini_preproc} shows the settings in question: 

\begin{lstlisting}[language=bash, numbers=none, caption=Paths and parameters for the Pre-processing phase in \textit{config.ini} file, label=lst:config_ini_preproc]	
	[PATHS]
	root_dir = /home/marcuzzo/UniVr/Tesi
	project_dir = %(root_dir)s/PLC-RE
	input_dataset_directory = %(root_dir)s/datasets_SWaT/2015
	net_csv_path = %(root_dir)s/datasets_SWaT/2015/Network_CSV
	
	[DEFAULTS]
	dataset_file = PLC_SWaT_Dataset.csv # Default output dataset
	granularity = 10  # slope granularity
	number_of_rows = 20000  # Seconds to consider
	skip_rows = 100000  # Skip seconds from beginning
\end{lstlisting}

Concurrently, the same options can be specified by the user via the command line of the new Python script (named \texttt{mergeDatasets.py} and contained in the directory \texttt{pre-processing} of the project) and will override the default ones found in \textit{config.ini}. These options are:

\begin{itemize}
	\item \textbf{-s} or \textbf{{-}{-}skiprows:} seconds to jump from the beginning of the file. This option is useful in case the system has an initial transient or to start the analysis from a certain point in the dataset
	
	\item \textbf{-n} or \textbf{{-}{-}nrows:} reference temporal period in seconds (rows) for the analysis from the beginning of the dataset or from the point specified in the \texttt{-s} option or in the corresponding setting in \textit{config.ini}.\newline
	This option makes a \textbf{selection} on the data of the dataset.
	
	\item \textbf{-p} or \textbf{{-}{-}plcs:} PLCs to be merged and enriched. The user can specify the desired PLCs by indicating the CSV file names of the associated datasets with no limitations on number.\newline
	This option makes a \textbf{projection} on the data of the dataset.
	
	\item \textbf{-d} or \textbf{{-}{-}directory:} performs the merge and enrichment of all CSV files contained in the directory specified by user, overriding the default setting in \textit{config.ini}. It is in fact the old functionality of the previous tool, maintained here to give the user more flexibility and convenience in case he wants to perform the analysis on the whole system. This is also the default behavior in case the \texttt{-p} option is not specified.
	
	\item \textbf{-o} or \textbf{{-}{-}output:} specifies the name of the file in which the obtained dataset will be saved. It must necessarily be a file in CSV format.
	
	\item \textbf{-g} or \textbf{{-}{-}granularity:} specifies a granularity that will be used to calculate the measurement slope during the dataset enrichment phase. We will discuss this later in Section \ref{subsec:4_dataset_enrichment}.
\end{itemize}
\vfill

\subsection{Dataset Enrichment}
\label{subsec:4_dataset_enrichment}

\subsection{Brief Analysis of the Subystem}
\label{subsec:4_brief_analysis}

\section{Phase 2: Graphs and Statistical Analysis}
\label{sec:improve_graphs}

\section{Phase 3: Invariants Analysis}
\label{sec:improve_invariants}

%\subsection{Automatic Detection of Actuators and Sensors}
\subsection{Invariants Generation}

\section{Phase 4: Businness Process Analysis}
\label{sec:improve_bpa}

%\section{Extra Information on the Physics}
%\label{sec:extra_info}

\vfill
\nolinenumbers