\chapter{State of the Art}
\label{state_of_art}
\linenumbers

\lettrine[lines=2]{I}{n traditional IT}, an attacker aims to understand the behavior of a program through various techniques so as to bring attacks aimed at changing its execution flow, functionalities or bypassing limits imposed by the licensing of such software. These attack techniques include a \textbf{preliminary study} of the program: a \textit{static analysis} (i.e., a preliminary analysis of the software without it running) and a \textit{dynamic analysis} (i.e., an analysis performed with the program running).\\
The result of these two preliminary investigation techniques is a \textbf{reverse engineering} of the software, which is useful for identifying any weaknesses or bugs and therefore planning an attack.

\bigskip
In the OT context, however, the concept of \textit{reverse engineering} is also associated with that of \textit{\textbf{process comprehension}}, a term coined by Green et al.'s \cite{proc_compr_green} to describe the understanding of the characteristics of the system and the physical elements of within it, that are responsible for its proper functioning.

\bigskip
Not much knowledge exists in the literature regarding the collection and analysis of information regarding the understanding and operation of an ICS: in Section \ref{sec:related_work} we will look at a quick overview of some of the existing literature on the subject and in the following sections we will focus in particular on one of the papers exposed.

\vfill

\section{Literature on Process Comprehension}
\label{sec:related_work}
\begin{description}
	\item[Keliris and Maniatikos] The first approach presented in this section is by Keliris and Maniatakos \cite{keliris_maniatakos}: they present a methodology for automating the reverse engineering of ICS binaries.
	
	\item[Yuan et al.] description
	
	\item[Feng et al.] description
	
	\item[Winnicki et al.] description
	
	\item[Green et al.] description
	
	\item[Pal et al.] description
	
	\item[Ceccato et al.] description
\end{description}

\section{Ceccato et al.’s methodology for analyzing water-tank systems}
\label{sec:ceccato_metodology}

\subsection{Scanning of the System and Graph Analysis}
\label{subsec:ceccato_graphanalysis}

\subsection{Businness Process Analysis}
\label{subsec:ceccato_businessprocess}

\subsection{Invariants Analysis}
\label{subsec:ceccato_invariants}

\subsection{Application to a Simulated Testbed}
\label{subsec:ceccato_testbed}

\subsection{Limitations}
\label{subsec:ceccato_limitations}

\nolinenumbers
\vfill