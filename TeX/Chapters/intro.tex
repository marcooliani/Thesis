\chapter{Introduction}
\label{intro}
\linenumbers

\lettrine[lines=2]{T}{he} advent of Industry 4.0 has sparked significant transformations in the realm of \textit{Industrial Control Systems} (ICS). In recent years, there has been a rapid expansion in the interconnectivity and convergence of IT (\textit{Information Technology}) and OT (\textit{Operational Technology}) systems. While this integration offers undeniable benefits in terms of operational efficiency and effectiveness of ICSs, it has also exposed these systems to the prevalent vulnerabilities commonly associated with IT environments. Consequently, there has been a parallel rise in \textbf{cyber-physical attacks}, which originates in the cyber environment and target the physical processes of these systems. These attacks aim to manipulate or disrupt the normal operation of ICSs, posing a considerable risk to their integrity and functionality.

\bigskip
\textit{Programmable Logic Controllers} (PLCs), which are essential elements of ICSs, play a vital role in managing electrical equipment like pumps, valves, centrifuges, and more. Serving as a bridge between the cyber and physical realms, PLCs possess a straightforward structure comprising a central processing module (CPU) and additional modules for handling physical inputs and outputs. The user program operates on the CPU and carries out \textit{scan cycles} that involve tasks such as gathering sensor data, executing controller code, and sending commands to actuator devices. Despite modern controllers incorporating security measures to upload authorized firmware, exploiting PLCs can still result in grave consequences for ICSs including physical damage, operational disruptions, environmental hazards, and even threats to human safety. These threats can originate from various sources, such as hackers, insider threats, or state-sponsored actors, and can have significant impacts on operational continuity, safety, and financial losses. The potential impact of an ICS breach necessitates robust security measures.

\bigskip
To execute a successful cyber-physical attack, an attacker must possess a comprehensive understanding of the target system's characteristics and behavior, commonly referred to as \textbf{\textit{process comprehension}}. This entails possessing knowledge about various aspects, including the \textbf{operational domain} (such as water distribution or power generation), the controllers used (such as PLCs, RTUs, etc.), the \textbf{network topology} associated with them, relevant \textbf{measurements} within the facility (such as pressure, temperature, etc.), as well as the exposed physical components like \textbf{sensors and actuators} that can be vulnerable to attacks. By attaining such insights, the attacker can effectively identify vulnerabilities, exploit weaknesses, and manipulate the system in a targeted and \textit{stealthy} manner. Moreover, to bypass \textit{intrusion detection systems} (IDSs), the attacker needs to possess a general understanding of the \textbf{physical invariants} of the system.

\bigskip
Understanding the process operations and functionalities of ICS from an attacker's perspective is a crucial aspect of conducting targeted cyber-attacks. By comprehending how the systems operate, their vulnerabilities, and potential impact, attackers can devise effective strategies to infiltrate and compromise the ICS environment. Here are some key points related to process comprehension of ICS from an attacker's viewpoint:

\begin{itemize}
	\item \textbf{System Architecture:} attackers aim to understand the architecture of the ICS, including the components involved, such as sensors, actuators, controllers, and human-machine interfaces. This knowledge helps them identify potential entry points and vulnerabilities within the system;
	
	\item \textbf{Industrial Protocols:} familiarity with industrial communication protocols, such as Modbus, DNP3, or OPC, is crucial for attackers. Understanding the protocols enables them to analyze the communication between different ICS components and potentially exploit vulnerabilities or manipulate the data flow; 
	
	\item \textbf{Operational Processes:} attackers seek to comprehend the operational processes and workflows within the targeted ICS. This includes understanding the sequence of actions, system states, and interactions between various components. Such knowledge enables attackers to identify critical points where disruptions or malicious actions can have significant consequences;
	
	\item \textbf{Process Dependencies:} attackers analyze the dependencies between different processes or systems within the ICS. They aim to identify dependencies that, if disrupted or compromised, could have a cascading effect on the overall operation. This helps them strategize targeted attacks for maximum impact;
	
	\item \textbf{Operational Technology (OT) and Information Technology (IT) Convergence:} attackers recognize the convergence of OT and IT systems within modern ICS environments. Understanding how the ICS interfaces with external IT networks and devices helps them identify potential attack vectors, such as vulnerable IT systems or insecure connections;
	
	\item \textbf{Attack Surface:} by evaluating the attack surface of the ICS, including network connectivity, remote access options, and third-party integrations, attackers can identify potential entry points and avenues for exploitation.
\end{itemize}

Attackers can acquire a good level of process comprehension of an ICS without employing \textit{Denial of Service} (DoS) techniques through several methods.  For instance, they can utilize \textbf{probing} techniques, which involve introducing slight perturbations to the system and observing its response and subsequent return to its original state. By analyzing these reactions, attackers can gain valuable insights into the system's behavior.

Another technique is the \textbf{\textit{Man-in-the-Middle} approach}, where attackers intercept and analyze network communications to gather information about the system. This allows them to conduct \textbf{reconnaissance} of the system while avoiding detection.

\textbf{\textit{Reverse engineering}} is another method attackers can employ. By scanning memory logs of network-exposed PLCs and analyzing network traffic related to the industrial protocols, they can approximate the physical system's model. This technique enables them to derive useful information without directly intervening in the physical system, as they can work offline using collected logs and analysis tools.

\section{Contribution}
\label{sec:1_contribution}

The original purpose of this thesis was to validate a specific methodology designed to attain \textit{process comprehension} in an Industrial Control System (ICS) when applied to a real-world scenario. This technology utilizes reverse engineering method and adopts a \textbf{black box approach}, involving dynamic analysis of both the physical process and network communications within the system. The primary goal of this methodology is to obtain a comprehensive understanding of the industrial process.

\bigskip
To accomplish this objective, a framework developed by the authors of the methodology is utilized. This framework incorporates a series of analysis steps that are employed to facilitate the application of the methodology. The entire framework and methodology are then applied to a specially implemented virtualized testbed, providing a controlled environment for testing and evaluation purposes and to facilitate the process of achieving process comprehension of an Industrial Control System.

As the thesis work advanced, it became progressively apparent that both the methodology and tools employed required revision and expansion. In response, an extensive revamping of the framework was conducted, aimed at enhancing and expanding its existing features while introducing new ones.

\bigskip
Therefore, the \textbf{primary contribution} of this thesis is to \textbf{enhance the original methodology} by refining the existing framework, reassessing the approach for each step of the analysis, and incorporating new features. The objective is to achieve a more comprehensive and thorough process comprehension of the industrial system under investigation. This entails expanding the methodology to gather a richer set of information, improving the accuracy of the analysis, and incorporating additional techniques to uncover hidden patterns and relationships within the system.

\bigskip
Furthermore, the enhanced methodology will be validated through the application to a different and larger case study, distinct from the virtualized testbed used previously. This real-world scenario will provide a robust validation of the methodology's effectiveness in a practical setting, ensuring its applicability and reliability in diverse industrial environments. 

\section{Outline}
\label{sec:1_outline}
\noindent The thesis is structured as follows:

\begin{description}
	\item [Chapter 2:] provides a background on Industrial Control Systems, (ICSs) describing their structure, components, and some of the network communication protocols used;
	\item [Chapter 3:] following an introductory section that provides a brief overview of the existing literature on process comprehension in industrial control systems, this chapter focuses on a specific paper that outlines a methodology to attaining process comprehension of an industrial system by employing dynamic blackbox analysis;
	\item [Chapter 4:] outlines a proposal to improve and extend the methodology outlined in the previous chapter;
	\item [Chapter 5:] presents the case study on which the proposed methodology will be applied;
	\item [Chapter 6:] shows how the proposed methodology is applied to the case study illustrated above;
	\item [Chapter 7:] outlines final conclusions and future work.
\end{description}

\vfill
\nolinenumbers