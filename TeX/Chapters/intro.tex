\chapter{Introduction}
\label{intro}
%\linenumbers

\lettrine[lines=2]{T}{he} advent of Industry 4.0 has sparked significant transformations in the realm of \textit{Industrial Control Systems} (ICS), physical and engineered systems whose operations are monitored, coordinated, controlled, and integrated by a computing and communication core \cite{ics_definition_giusta}. They are used to control industrial processes such as manufacturing, product handling, production, and distribution \cite{ics_definition}. In recent years, there has been a rapid expansion in the interconnectivity and convergence of IT (\textit{Information Technology}) and OT (\textit{Operational Technology}) systems. While this integration offers undeniable benefits in terms of operational efficiency and effectiveness of ICSs, it has also exposed these systems to the prevalent vulnerabilities commonly associated with IT environments. Consequently, there has been a parallel rise in \textbf{cyber-physical attacks}, which originates in the cyber environment and target the physical processes of these systems. These attacks aim to manipulate or disrupt the normal operation of ICSs, posing a considerable risk to their integrity and functionality.

\paragraph{Contextualization}
\label{par:1_context}
As mentioned earlier, the distinction between the IT (\textit{Information Technology}) and OT (\textit{Operational Technology}) components within an ICS is becoming increasingly blurred. The IT part pertains to the \textbf{cyber} aspects of an industrial system, facilitating the collaboration and communication of information processing technologies as well as technologies for monitoring, control, and maintenance purposes \cite{tesi_phd_norvegese}.

On the other hand, the OT part primarily focuses on the aspects related to the \textbf{physical system}. The OT networks of ICSs typically encompass devices, systems, networks, and controllers used for operating and automating industrial processes. These networks consist of \textit{field devices}, such as sensors and actuators, which monitor and control the progression of a physical process over time. They also include \textit{Programmable Logic Controllers} (PLCs) responsible for controlling the field devices, as well as one or more \textit{Human Machine Interfaces} (HMIs) that enable human operators to interact with the PLCs and display status information and historical data collected by the field devices within the ICS environment.

OT networks encompass two primary sub-networks: the \textit{supervisory control network}, which connects the PLCs and HMIs, and the \textit{field communications network}, which establishes connections between the PLCs and the associated field devices.

\bigskip
\textit{Programmable Logic Controllers} (PLCs), which are essential elements of ICSs, play a vital role in managing electrical equipment like pumps, valves, centrifuges, and more. Serving as a bridge between the cyber and physical realms, PLCs possess a straightforward structure comprising a central processing module (CPU) and additional modules for handling physical inputs and outputs. The user program operates on the CPU and carries out \textit{scan cycles} that involve tasks such as gathering sensor data, executing controller code, and sending commands to actuator devices. Despite modern controllers incorporating security measures to upload authorized firmware, exploiting PLCs can still result in grave consequences for ICSs including physical damage, operational disruptions, environmental hazards, and even threats to human safety. These threats can originate from various sources, such as hackers, insider threats, or state-sponsored actors, and can have significant impacts on operational continuity, safety, and financial losses. The potential impact of an ICS breach necessitates robust security measures.

\bigskip
The increasing convergence of the IT and OT components in an ICS has led to a transition from serial-based communication protocols to IP-based protocols. These protocols, known as \textbf{industrial protocols}, have become prevalent in modern industrial systems. Examples of industrial protocols include Modbus \cite{Modbus_definition}, DNP3 \cite{dnp3}, EtherNet/IP \cite{enip_pdf2}, OPC UA \cite{opc}, and S7comm \cite{s7comm} (which is a proprietary and closed protocol).

The adoption of IP-based protocols introduces a shared vulnerability between IT and OT networks. When OT networks utilize exposed networks like the Internet for communication, they become susceptible to the same security risks as IT networks. Similar to PLCs, breaches within the communication network can have significant consequences. Therefore, it is crucial to implement robust security measures to prevent potential cyber-physical attacks and ensure the integrity and security of the entire system.

\paragraph{Open Problems}
\label{par:1_open_problems}
To execute a successful cyber-physical attack, an attacker must possess a comprehensive understanding of the target system's characteristics and behavior, commonly referred to as \textbf{\textit{process comprehension}} \cite{green_et_al}. This entails possessing knowledge about various aspects, including the \textbf{operational domain} (such as water distribution or power generation), the controllers used (such as PLCs, RTUs, etc.), the \textbf{network topology} associated with them, relevant \textbf{measurements} within the facility (such as pressure, temperature, etc.), as well as the exposed physical components like \textbf{sensors and actuators} that can be vulnerable to attacks. By attaining such insights, the attacker can effectively identify vulnerabilities, exploit weaknesses, and manipulate the system in a targeted and \textit{stealthy} manner. Moreover, to bypass certain types of \textit{intrusion detection systems} (IDSs), the attacker needs to possess a general understanding of the \textbf{physical invariants} of the system.

\bigskip
Understanding the process operations and functionalities of ICS from an attacker's perspective is a crucial aspect of conducting targeted cyber-attacks. By comprehending how the systems operate, their vulnerabilities, and potential impact, attackers can devise effective strategies to infiltrate and compromise the ICS environment. Here are some key points related to process comprehension of ICS from an attacker's viewpoint:

\begin{itemize}
	\item \textbf{System Architecture:} attackers aim to understand the architecture of the ICS, including the components involved, such as sensors, actuators, controllers, and human-machine interfaces. This knowledge helps them identify potential entry points and vulnerabilities within the system;
	
	\item \textbf{Industrial Protocols:} familiarity with industrial communication protocols, such as Modbus \cite{Modbus_definition}, DNP3 \cite{dnp3}, or EtherNet/IP - CIP \cite{enip_pdf}, is crucial for attackers. Understanding the protocols enables them to analyze the communication between different ICS components and potentially exploit vulnerabilities or manipulate the data flow; 
	
	\item \textbf{Operational Processes:} attackers seek to comprehend the operational processes and workflows within the targeted ICS. This includes understanding the sequence of actions, system states, and interactions between various components. Such knowledge enables attackers to identify critical points where disruptions or malicious actions can have significant consequences;
	
	\item \textbf{Process Dependencies:} attackers analyze the dependencies between different physical processes or systems within the ICS. They aim to identify dependencies that, if disrupted or compromised, could have a cascading effect on the overall operation. This helps them strategize targeted attacks for maximum impact;
	
	%\item \textbf{Operational Technology (OT) and Information Technology (IT) Convergence:} attackers recognize the convergence of OT and IT systems within modern ICS environments. Understanding how the ICS interfaces with external IT networks and devices helps them identify potential attack vectors, such as vulnerable IT systems or insecure connections;
	
	\item \textbf{Attack Surface:} by evaluating the attack surface of the ICS, including network connectivity, remote access options, and third-party integrations, attackers can identify potential entry points and avenues for exploitation;
	
	\item \textbf{Attack Goals:} attackers define their specific goals, whether it be disruption of operations, theft of sensitive data, sabotage, or any other malicious intent. Understanding the desired outcomes helps them tailor their attack strategies and prioritize their actions within the ICS environment.
	
\end{itemize}

Attackers can acquire a good level of process comprehension of an ICS %without employing \textit{Denial of Service} (DoS) techniques 
through several methods.  For instance, they can utilize \textbf{probing} techniques \cite{winnicki_et_al}, which involve introducing slight perturbations to the system and observing its response and subsequent return to its original state. By analyzing these reactions, attackers can gain valuable insights into the system's behavior.

Another technique is the \textbf{\textit{Man-in-the-Middle} approach} \cite{green_et_al}, where attackers intercept and analyze network communications to gather information about the system. This allows them to conduct \textbf{reconnaissance} of the system while avoiding detection.

\textbf{\textit{Reverse engineering}} \cite{ceccato}\cite{keliris_maniatakos} is another method attackers can employ. By scanning memory logs of network-exposed PLCs and analyzing network traffic related to the industrial protocols, they can approximate the physical system's model. This technique enables them to derive useful information without directly intervening in the physical system, as they can work offline using collected logs and analysis tools.

\bigskip
By leveraging these techniques, attackers can execute targeted and precise attacks on the system, avoiding the need for \textit{Denial of Service} (DoS) tactics. This enables them to focus on exploiting specific vulnerabilities and achieving their objectives with greater precision and effectiveness.

\section*{Contribution}
\label{sec:1_contribution}

The primary objective of this thesis is to conduct an analysis of Ceccato et al.'s reverse engineering-based methodology \cite{ceccato}. The aim is to examine its strengths and limitations and apply it to a real-world case study.

\bigskip
Ceccato et al.'s methodology relies on a specialized framework that performs the analysis of an ICS through multiple phases. It begins with the collection of system-related data by scanning the logs of exposed PLCs for information on the physical system and analyzing network traffic related to the industrial protocols used for communications. Subsequent phases involve enriching the obtained data for the physical system, conducting graphical-statistical analysis of PLC logs, inferring system invariants using Daikon, and understanding the overall behavior of the PLCs in conjunction with network traffic data through a business process analysis. Ceccato et al. utilize a virtualized testbed, simulating a water treatment plant, with communication occurring through the Modbus protocol.

\bigskip
Ceccato et al.'s framework has \textbf{limitations} that hinder its use on systems other than their specific testbed. One limitation is that the testbed itself is overly simplified compared to real-world scenarios, with only three stages and few components, resulting in a limited number of registers. Moreover, the virtualization does not account for the real physics of water and its oscillations. Another limitation is that their Graphical Analysis can display only one register at a time, making it impossible to have an overall view of the system or capture relationships between registers. Additionally, the output produced by the external tool Daikon \cite{daikon_site} for invariant analysis is difficult to interpret due to its poorly readable format.

Overall, the Ceccato et al. framework lacks flexibility in analyzing different elements and relies heavily on ad hoc solutions for their specific testbed and the Modbus protocol. This can lead to limited and inaccurate results when applied to other industrial systems, and in some cases, it may even be impossible to perform the analysis.

\bigskip
The objective of this thesis is to address these limitations identified in Ceccato et al.'s framework by improving and expanding upon the original framework. This involves a complete overhaul and rewriting of the code to introduce previously absent features and enhance existing ones. The framework has been made independent of the system being analyzed, allowing for customization through a configuration file. Furthermore, enhancements and expansions have been introduced across all analysis phases.

\bigskip
For instance, \textit{Network Analysis} has been added to discover \textbf{network topology and device communications}, regardless of the industrial protocol used. Enhancements have also been made to other phases, including improvements to the \textbf{command-line interface} for increased flexibility and user-friendliness, as well as the ability to extend the scope of the analysis.

The \textit{Pre-processing} phase, responsible for enriching data obtained from physical system scans, now allows the association of additional fields such as slope with specific registers or groups of registers. This provides a more accurate dataset and facilitates operations in other phases. Additionally, a new \textbf{preliminary analysis phase} has been introduced, automating the recognition of probable measurements and actuators, and analyzing various properties related to them.

The \textit{Graphical Analysis} component has undergone significant improvements to enhance its effectiveness. \textbf{Subplots} now enable the visualization of registers as a whole, facilitating the identification of relationships between them. Users can select specific registers to view and zoom in on specific areas of the graphs without losing information.

\textit{Invariant Analysis} has also been revised to \textbf{improve the output generated by Daikon}, making it more concise and readable for easier identification of invariants. \textbf{Two semi-automatic analyses} based on individual actuator states and current system states have been introduced as well.

Finally, the \textit{Business Process Analysis} has been revamped to extract as much information as possible from the physical system, including actual system states. This information is then integrated with network data using a newly developed solution instead of relying on an external tool as in the previous version.

%%
\bigskip
The new framework will be tested on a \textbf{more complex case study}, namely the iTrust SWaT system \cite{swat_home}, which is an actual water treatment system. Compared to Ceccato et al.'s testbed, the iTrust SWaT system involves more PLCs, registers, and components, presenting additional challenges. For instance, fluctuations in water levels within the tanks may introduce data perturbations that need to be mitigated to ensure accurate results aligned with reality.

Furthermore, the iTrust SWaT system employs the CIP over EtherNet/IP protocol, which differs significantly from Modbus. Consequently, different implementation approaches are required when considering network data within the various analysis steps, as compared to Ceccato et al.'s methodology.

\section*{Outline}
\label{sec:1_outline}
\noindent The thesis is structured as follows:

\begin{description}
	\item [Chapter 2:] provides a background on Industrial Control Systems, (ICSs) describing their structure, components, and some of the network communication protocols used;
	
	\item [Chapter 3:] after an introductory section that provides a brief overview of the existing literature on process comprehension in industrial control systems, this chapter focuses on a specific paper that outlines a methodology to attaining process comprehension of an industrial system by employing dynamic blackbox analysis;
	
	\item [Chapter 4:] defines a proposal to improve and extend the methodology outlined in the previous chapter;
	
	\item [Chapter 5:] presents the case study on which the proposed methodology will be applied;
	
	\item [Chapter 6:] shows how the proposed methodology is applied to the case study illustrated above;
	
	\item [Chapter 7:] outlines final conclusions and future work.
\end{description}

\vfill
%\nolinenumbers