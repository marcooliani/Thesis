\chapter{Introduction}
\label{intro}
\linenumbers

\lettrine[lines=2]{L}{orem} ipsum dolor bla bla bla. Ma dove metto l'abstract? Prova di interlinea che direi posso anche andare bene, ma bisogna poi vedere il tutto come si incastra alla fine, in modo da ottenere un bel risultato alla vista. 

\section{Contribution}
\label{sec:contribution}

The primary objective of this thesis was originally to validate a specific methodology designed to attain process comprehension in an Industrial Control System (ICS) when applied to a real-world scenario. This methodology, which adopts a blackbox approach, involves dynamic analysis of the physical process and network communications of the system. The ultimate goal of this methodology is to accurately derive a comprehensive understanding of the industrial process.

\bigskip
To accomplish this objective, a framework developed by the authors of the methodology is utilized. This framework incorporates a series of analysis steps that are employed to facilitate the application of the methodology. The entire framework and methodology are then applied to a specially implemented virtualized testbed, providing a controlled environment for testing and evaluation purposes and to facilitate the process of achieving process comprehension of an Industrial Control System.\newline
As the thesis progressed, it quickly became evident that both the framework and the methodology employed had certain limitations. In response, an extensive restructuring of the framework was conducted, aimed at enhancing and expanding its existing features while introducing new ones.

\bigskip
Hence, the main contribution of the thesis is now to enhance the original methodology by refining the framework and reassessing the approach for each analysis step. The aim is to achieve a more complete and exhaustive process comprehension. Subsequently, the improved methodology will be tested on a different and more complex case study, distinct from the virtualized testbed, to validate its effectiveness in a real-world scenario.

\section{Outline}
\label{sec:outline}
\noindent The thesis is structured as follows:

\begin{description}
	\item [Chapter 2:] provides a background on Industrial Control Systems, (ICSs) describing their structure, components, and some of the network communication protocols used;
	\item [Chapter 3:] following an introductory section that provides a brief overview of the existing literature on process comprehension in industrial control systems, this chapter focuses on a specific paper that outlines a methodology to attaining process comprehension of an industrial system by employing dynamic blackbox analysis;
	\item [Chapter 4:] outlines a proposal to improve and extend the methodology outlined in the previous chapter;
	\item [Chapter 5:] presents the case study on which the proposed methodology will be applied;
	\item [Chapter 6:] shows how the proposed methodology is applied to the case study illustrated above;
	\item [Chapter 7:] outlines final conclusions and future work.
\end{description}

\vfill
\nolinenumbers