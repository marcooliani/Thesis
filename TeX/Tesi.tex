%% Definisco la classe del documento: va specificato il formato della carta,
%% le dimensioni del carattere, di cominciare i nuovi capitoli a destra e
%% fronte-retro
\documentclass[a4paper,12pt,openright,twoside]{book}

%% Imposto l'italiano
\usepackage[italian]{babel}

%% Imposto l'encoding dei caratteri. Specificando utf8 posso mettere
%% direttamente le accentate senza errori
\usepackage[utf8]{inputenc}

%% Imposto l'indentazione per ogni paragrafo
\usepackage{indentfirst}
\usepackage{fancyhdr}

%% Questo per aforismi e citazioni
\usepackage{epigraph}

%% Per inserire gli url all'interno del testo
\usepackage[hidelinks]{hyperref}

%% Per inserire codici sorgenti
\usepackage{listings}
\usepackage{xcolor}
\definecolor{light-gray}{gray}{0.80}
\addto\captionsitalian{\renewcommand{\lstlistingname}{Esempio}}


%% Elenchi puntati e affini
\usepackage{enumitem}

%% Interlinea a 1,5
\linespread{1.5}

%% Per includere le immagini: package e path dove andare a richiamarle per nome
\usepackage{graphicx}
\graphicspath{{./images/}}

%% Questo serve per includere le immagini dove voglio io e non dove
%% pare all'editor!
\usepackage{float}

%% Fa sì che le tabelle possano essere divise tra più pagine
\usepackage{longtable}

%% Imposto l'interlinea al volo
\usepackage{setspace}

%% Formattazioni varie
\pagestyle{fancy}\addtolength{\headwidth}{20pt}
\renewcommand{\chaptermark}[1]{\markboth{\thechapter.\ #1}{}}
\renewcommand{\sectionmark}[1]{\markright{\thesection \ #1}{}}
\cfoot{}

\rhead[\fancyplain{}{\bfseries\leftmark}]{\fancyplain{}{\bfseries\thepage}}
\lhead[\fancyplain{}{\bfseries\thepage}]{\fancyplain{}{\bfseries\rightmark}}

%% Inizio documento
\begin{document}
	
	%% Prova di frontespizio
	\begin{titlepage}
		\begin{center}
			\thispagestyle{empty}
			
			% logo
			%\includegraphics[width=0.20\textwidth]{logo_univr.png}~\\[0.5cm]
			
			% Upper part of the page. The '~' is needed because \\
			% only works if a paragraph has started.
			
			\textsc{\LARGE Università degli Studi di Verona}\\[1.0cm]
			
			\textsc{\large Corso di Laurea Magistrale in \\ 
				INGEGNERIA E SCIENZE INFORMATICHE}\\[2.0cm]
			
			% Title
			
			{ \Large \bfseries [PROVVISORIO] \\
				REVERSE ENGINEERING DI CPS \\ 
				CON APPROCCIO BLACKBOX \\[4.0cm] }
			
			% Author and supervisor
			\begin{minipage}{0.4\textwidth}
				\begin{flushleft} \large
					\emph{Relatore:}\\
					Prof. Massimo \textsc{Merro}
				\end{flushleft}
			\end{minipage}
			\begin{minipage}{0.4\textwidth}
				\begin{flushright} \large
					\emph{Laureando:} \\
					Marco \textsc{Oliani} 
				\end{flushright}
			\end{minipage}
			
			\vfill
			
			% Bottom of the page -> è la data
			{\large Anno Accademico 2022/2023}
			
		\end{center}
	\end{titlepage}

\thispagestyle{empty}

%% Aforisma
\begin{titlepage}
	\thispagestyle{empty}
	\begin{flushright}
		\vspace*{40 mm}
		\small{\emph{``Per fare una cosa ci vuole sempre più tempo di quanto\\
				si pensi, anche tenendo conto della Legge di Hofstadter''}\\(Legge di Hofstadter)}
	\end{flushright}
	
	\vfill
\end{titlepage}
\thispagestyle{empty}

%% Faccio partire la numerazione normale DOPO l'indice: lo stesso indice 
%% verrà numerato in numeri romani
\frontmatter
\tableofcontents
\mainmatter

%%%%%%%%%%%%%%%%%%%%%%%%%%%%%%%%%
%       Inizio TESI             %
%%%%%%%%%%%%%%%%%%%%%%%%%%%%%%%%%


%%%%%%%%%%%%%%%%%%%%%%%%%%%%%%%%%
%		Fine TESI				%
%%%%%%%%%%%%%%%%%%%%%%%%%%%%%%%%%
\end{document}