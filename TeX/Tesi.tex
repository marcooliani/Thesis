%% Definisco la classe del documento: va specificato il formato della carta,
%% le dimensioni del carattere, di cominciare i nuovi capitoli a destra e
%% fronte-retro
\documentclass[a4paper,12pt,openright,twoside,titlepage]{book}

%% Imposto l'italiano
\usepackage[english,italian]{babel}

%% Imposto l'encoding dei caratteri. Specificando utf8 posso mettere
%% direttamente le accentate senza errori
\usepackage[utf8]{inputenc}

\usepackage{fancyhdr}

%% Formato del titolo dei capitoli
\usepackage[Lenny]{fncychap}
\ChNumVar{\fontsize{76}{80}\usefont{OT1}{pzc}{m}{n}\selectfont}
\ChTitleVar{\centering\Large\sffamily\bfseries}

%% Font (sceglierne uno)
%\usepackage{tgpagella} % TeX Gyre Pagella
%\usepackage{charter} % PSNFSS Font, in every TeX distribution

\usepackage{lmodern} % monospace font
\usepackage[T1]{fontenc}
\renewcommand{\ttdefault}{lmtt} % MONO Latin Modern Font % T1 encoding of cmtt font style  
\renewcommand{\rmdefault}{lmr} % SERIF Latin Modern Font % T1 encoding of cmr font style
\renewcommand{\sfdefault}{lmss} % SANS Latin Modern Font % T1 encoding of cmss font style

%% Imposto l'indentazione per ogni paragrafo
\usepackage{indentfirst}

%% Lettera iniziale del capitolo su più righe
\usepackage{lettrine}

%% Interlinea a 1,5
\usepackage{setspace}
\onehalfspacing % interlinea 1.5
%\doublespacing % interlinea 2

%% Questo per aforismi e citazioni
\usepackage{epigraph}

%% Per inserire gli url all'interno del testo
\usepackage[hidelinks]{hyperref}

%% Per inserire codici sorgenti
\usepackage{listings}
\usepackage{xcolor}
\definecolor{light-gray}{gray}{0.80}
\addto\captionsitalian{\renewcommand{\lstlistingname}{Esempio}}

%% Elenchi puntati e affini
\usepackage{enumitem}

%% Per includere le immagini: package e path dove andare a richiamarle per nome
\usepackage{graphicx}
\graphicspath{{./images/}}

%% Questo serve per includere le immagini dove voglio io e non dove
%% pare all'editor!
\usepackage{float}

%% Fa sì che le tabelle possano essere divise tra più pagine
\usepackage{longtable}

%% Formattazioni varie
\pagestyle{fancy}\addtolength{\headwidth}{20pt}
\renewcommand{\chaptermark}[1]{\markboth{\thechapter.\ #1}{}}
\renewcommand{\sectionmark}[1]{\markright{\thesection \ #1}{}}
\cfoot{}

\rhead[\fancyplain{}{\bfseries\leftmark}]{\fancyplain{}{\bfseries\thepage}}
\lhead[\fancyplain{}{\bfseries\thepage}]{\fancyplain{}{\bfseries\rightmark}}

%% Abstract
\newcommand{\fncyblank}{\fancyhf{}}
\newenvironment{abstract}%
{\cleardoublepage\fncyblank\null\vfill\begin{center}%
	\bfseries\abstractname\end{center}}%
{\vfill\null}

%######
%% Inizio documento
%######
\begin{document}

%###### Frontespizio ######%
\begin{titlepage}
	\begin{center}
		\thispagestyle{empty}
		
		% logo
		%\includegraphics[width=0.15\textwidth]{logo_univr.png}~\\[0.5cm]
		
		% Upper part of the page. The '~' is needed because \\
		% only works if a paragraph has started.
		
		\textsc{\LARGE Università degli Studi di Verona}\\
		%\textsc{Facolt\`a di Scienze Matematiche, Fisiche e Naturali}\\[1.0cm]
		\textsc{Dipartimento di Informatica}\\[1.0cm]
		\textsc{\large Corso di Laurea Magistrale in \\ 
			INGEGNERIA E SCIENZE INFORMATICHE}\\[2.0cm]
		
		\large Tesi di Laurea Magistrale \\[1.5cm]
		% Title
		
		{ \Large \bfseries [PROVVISORIO] \\
			REVERSE ENGINEERING DI CPS \\ 
			CON APPROCCIO BLACKBOX \\[4.0cm] }
		
		% Author and supervisor
		\begin{minipage}[t]{0.4\textwidth}
			\begin{flushleft} \large
				\emph{Relatore:}\\
				Chiar.mo Prof. \\
				\textbf{Massimo \textsc{Merro}}
			\end{flushleft}
		\end{minipage}
		\begin{minipage}[t]{0.4\textwidth}\raggedleft
			\begin{flushright} \large
				\emph{Laureando:} \\
				%Matr. VR457249 \\
				\textbf{Marco \textsc{Oliani}} 
			\end{flushright}
		\end{minipage}
		
		\vfill
		
		% Bottom of the page -> è la data
		{\large Anno Accademico 2022/2023}
		
	\end{center}
\end{titlepage}

\thispagestyle{empty}

%###### Aforisma ######%
\begin{titlepage}
	\thispagestyle{empty}
	\begin{flushright}
		\vspace*{40 mm}
		\small{\emph{``Gallina vecchia fa buon brodo''}\\(Valentino Rossi)}
	\end{flushright}	
	\vfill
\end{titlepage}
\thispagestyle{empty}

%###### ABSTRACT ######%
\begin{titlepage}
	\begin{abstract}
		Bla bla bla
		\vfill
	\end{abstract}
\end{titlepage}
\thispagestyle{empty}

%% Faccio partire la numerazione normale DOPO l'indice: lo stesso indice 
%% verrà numerato in numeri romani
\frontmatter
\tableofcontents
\mainmatter

%%%%%%%%%%%%%%%%%%%%%%%%%%%%%%%%%
%       Inizio TESI             %
%%%%%%%%%%%%%%%%%%%%%%%%%%%%%%%%%


%###### CAPITOLO 1 ######%
\chapter{Introduzione}
\lettrine[lines=2]{L}{orem} Bla bla bla. Ma dove metto l'abstract? Prova di interlinea che direi posso anche andare bene, ma bisogna poi vedere il tutto come si incastra alla fine, in modo da ottenere un bel risultato alla vista. Sperando di non esagerare con gli abbellimenti, per evitare che qualcuno leggendo si incazzi e mi faccia riscrivere tutto, che non ne ho minimamente voglia...

%###### CAPITOLO 2 ######%
\chapter{Altro Capitolo}
Ri-bla


%###### BIBLIOGRAFIA ######%
\begin{thebibliography}{99}
	\addcontentsline{toc}{chapter}{Bibliografia}
	
\end{thebibliography}

%%%%%%%%%%%%%%%%%%%%%%%%%%%%%%%%%
%		Fine TESI				%
%%%%%%%%%%%%%%%%%%%%%%%%%%%%%%%%%
\end{document}