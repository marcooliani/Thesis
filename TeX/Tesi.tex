%% Definisco la classe del documento: va specificato il formato della carta,
%% le dimensioni del carattere, di cominciare i nuovi capitoli a destra e
%% fronte-retro
\documentclass[a4paper,12pt,openright,twoside,titlepage]{book}

%% Imposto l'italiano
\usepackage[english,italian]{babel}

%% Imposto l'encoding dei caratteri. Specificando utf8 posso mettere
%% direttamente le accentate senza errori
\usepackage[utf8]{inputenc}
\usepackage[T1]{fontenc}

\usepackage{fancyhdr}

%% Formato del titolo dei capitoli
\usepackage[Lenny]{fncychap}
\ChNumVar{\fontsize{76}{80}\usefont{OT1}{pzc}{m}{n}\selectfont}
\ChTitleVar{\centering\Large\sffamily\bfseries}

%% Font (sceglierne uno)
%\usepackage{tgpagella} % TeX Gyre Pagella
%\usepackage{charter} % PSNFSS Font, in every TeX distribution
\usepackage{mathpazo} % Use the Palatino font by default

%\usepackage{lmodern} % monospace font
%\renewcommand{\ttdefault}{lmtt} % MONO Latin Modern Font % T1 encoding of cmtt font style  
%\renewcommand{\rmdefault}{lmr} % SERIF Latin Modern Font % T1 encoding of cmr font style
%\renewcommand{\sfdefault}{lmss} % SANS Latin Modern Font % T1 encoding of cmss font style

%% Imposto l'indentazione per ogni paragrafo
\usepackage{indentfirst}

%% Lettera iniziale del capitolo su più righe
\usepackage{lettrine}

%% Interlinea a 1,5
\usepackage{setspace}
\onehalfspacing % interlinea 1.5
%\singlespacing %interlinea 1
%\doublespacing % interlinea 2

%% Questo per aforismi e citazioni
\usepackage{epigraph}

%% Per inserire gli url all'interno del testo
\usepackage[hidelinks]{hyperref}

%% Per inserire codici sorgenti
\usepackage{listings}
\usepackage{xcolor}
\definecolor{light-gray}{gray}{0.80}
\addto\captionsitalian{\renewcommand{\lstlistingname}{Esempio}}

%% Elenchi puntati e affini
\usepackage{enumitem}

%% Per includere le immagini: package e path dove andare a richiamarle per nome
\usepackage{graphicx}
\graphicspath{{./figures/}}

%% Questo serve per includere le immagini dove voglio io e non dove
%% pare all'editor!
\usepackage{float}

%% Fa sì che le tabelle possano essere divise tra più pagine
\usepackage{longtable}

%% Bibliografia
\usepackage[sorting=none,style=numeric-comp,backend=bibtex]{biblatex}
\addbibresource{biblio/bibliography.bib} % Database con le citazioni

%%
%% Titolo della tesi
%%
\newcommand{\thesistitle}{Reverse Engineering di CPS con approccio blackbox}

%% Formattazioni varie
\pagestyle{fancy}\addtolength{\headwidth}{20pt}
\renewcommand{\chaptermark}[1]{\markboth{\thechapter.\ #1}{}}
\renewcommand{\sectionmark}[1]{\markright{\thesection \ #1}{}}
\cfoot{}

%%Testatine
%\rhead[\fancyplain{}{\bfseries\leftmark}]{\fancyplain{}{\bfseries\thepage}}
%\lhead[\fancyplain{}{\bfseries\thepage}]{\fancyplain{}{\bfseries\rightmark}}

%Pagine pari
\fancyhead[RE]{\bfseries\thesistitle}
\fancyhead[LE]{\bfseries\thepage}
\fancyhead[CE]{}
%\fancyfoot[LE]{\bfseries\thepage}
%\fancyfoot[CE]{}
%\fancyfoot[LE]{\textit{thesistitle}}

%Pagine dispari
\fancyhead[LO]{\bfseries\leftmark}
\fancyhead[RO]{\bfseries\thepage}
\fancyhead[CO]{}
%\fancyfoot[LO]{\bfseries\leftmark}
%\fancyfoot[CO]{}
%\fancyfoot[RO]{\textit{thesistitle}}
%\renewcommand{\footrulewidth}{0.4pt}

\fancypagestyle{plain}{%
	\fancyhf{}
	\fancyfoot[RO]{}
	\fancyfoot[LO]{}
	\renewcommand{\headrulewidth}{0pt}
	\renewcommand{\footrulewidth}{0pt}}

%% Abstract
\newcommand{\fncyblank}{\fancyhf{}}
\newenvironment{abstract}%
{\cleardoublepage\fncyblank\null\vfill\begin{center}%
	\bfseries\abstractname\end{center}}%
{\vfill\null}

%######
%% Inizio documento
%######
\begin{document}

%###### Frontespizio ######%
\begin{titlepage}
	\begin{center}
		\thispagestyle{empty}
		
		% logo
		%\includegraphics[width=0.15\textwidth]{logo_univr.png}~\\[0.5cm]
		
		% Upper part of the page. The '~' is needed because \\
		% only works if a paragraph has started.
		
		\textsc{\LARGE Università degli Studi di Verona}\\
		%\textsc{Facolt\`a di Scienze Matematiche, Fisiche e Naturali}\\[1.0cm]
		\textsc{Dipartimento di Informatica}\\[1.0cm]
		\textsc{\large Corso di Laurea Magistrale in \\ 
			INGEGNERIA E SCIENZE INFORMATICHE}\\[1.5cm]
		
		\large Tesi di Laurea Magistrale \\[1.0cm]
		% Title
		
		\hrulefill
		
		{ \Large\bfseries [PROVVISORIO] \\
			REVERSE ENGINEERING DI CPS \\ 
			CON APPROCCIO BLACKBOX \\ }
		
		\hrulefill \\[2.0cm]
		
		% Author and supervisor
		\begin{minipage}[t]{0.4\textwidth}
			\begin{flushleft} \large
				\emph{Relatore:}\\
				%Chiar.mo Prof. \\
				Prof. \textbf{Massimo \textsc{Merro}}
			\end{flushleft}
			\begin{flushleft} \large
				\emph{Correlatore:}\\
				Prof. \textbf{Severus \textsc{Piton}}
			\end{flushleft}
		\end{minipage}
		\begin{minipage}[t]{0.4\textwidth}\raggedleft
			\begin{flushright} \large
				\emph{Laureando:} \\
				\textbf{Marco \textsc{Oliani}} \\
				Matr. VR457249
			\end{flushright}
		\end{minipage}
		
		\vfill
		
		% Bottom of the page -> è la data
		{\large Anno Accademico 2022/2023}
		
	\end{center}
\end{titlepage}

\thispagestyle{empty}

%###### Aforisma ######%
\begin{titlepage}
	\thispagestyle{empty}
	\begin{flushright}
		\vspace*{40 mm}
		\small{\emph{``Gallina vecchia fa buon brodo''}\\(Valentino Rossi)}
	\end{flushright}	
	\vfill
\end{titlepage}
\thispagestyle{empty}

%###### ABSTRACT ######%
\begin{titlepage}
	\begin{abstract}
		Bla bla bla
	\end{abstract}
	\vfill
\end{titlepage}
\thispagestyle{empty}

%% Faccio partire la numerazione normale DOPO l'indice: lo stesso indice 
%% verrà numerato in numeri romani
\frontmatter
\tableofcontents
\mainmatter

%%%%%%%%%%%%%%%%%%%%%%%%%%%%%%%%%
%       Inizio TESI             %
%%%%%%%%%%%%%%%%%%%%%%%%%%%%%%%%%

%###### CAPITOLO 1 ######%
\chapter{Introduction}
\label{intro}
\linenumbers

\lettrine[lines=2]{T}{he} advent of Industry 4.0 has sparked significant transformations in the realm of Industrial Control Systems (ICS). In recent years, there has been a rapid expansion in the interconnectivity and convergence of IT (\textit{Information Technology}) and OT (\textit{Operational Technology}) systems. While this integration offers undeniable benefits in terms of operational efficiency and effectiveness of ICSs, it has also exposed these systems to the prevalent vulnerabilities commonly associated with IT environments. Consequently, there has been a parallel rise in \textbf{cyber-physical attacks}, which originates in the cyber environment and target the physical processes of these systems. These attacks aim to manipulate or disrupt the normal operation of ICSs, posing a considerable risk to their integrity and functionality.

\bigskip
To execute a successful cyber-physical attack, an attacker must possess a comprehensive understanding of the target system's characteristics and behavior, commonly referred to as \textbf{\textit{process comprehension}}. This entails acquiring detailed knowledge about the underlying physical processes, control mechanisms, communication protocols, and interdependencies within the system. By attaining such insights, the attacker can effectively identify vulnerabilities, exploit weaknesses, and manipulate the system in a targeted and stealthy manner.

\vfill

\section{Contribution}
\label{sec:1_contribution}

The original purpose of this thesis was to validate a specific methodology designed to attain \textit{process comprehension} in an Industrial Control System (ICS) when applied to a real-world scenario. This methodology, which adopts a \textbf{blackbox approach}, involves dynamic analysis of the physical process and network communications of the system. The primary goal of this methodology is to obtain a comprehensive understanding of the industrial process.

\bigskip
To accomplish this objective, a framework developed by the authors of the methodology is utilized. This framework incorporates a series of analysis steps that are employed to facilitate the application of the methodology. The entire framework and methodology are then applied to a specially implemented virtualized testbed, providing a controlled environment for testing and evaluation purposes and to facilitate the process of achieving process comprehension of an Industrial Control System.

As the thesis work advanced, it became progressively apparent that both the methodology and tools employed required revision and expansion. In response, an extensive revamping of the framework was conducted, aimed at enhancing and expanding its existing features while introducing new ones.

\bigskip
Therefore, the \textbf{primary contribution} of this thesis is to \textbf{enhance the original methodology} by refining the existing framework, reassessing the approach for each step of the analysis, and incorporating new features. The objective is to achieve a more comprehensive and thorough process comprehension of the industrial system under investigation. This entails expanding the methodology to gather a richer set of information, improving the accuracy of the analysis, and incorporating additional techniques to uncover hidden patterns and relationships within the system.

\bigskip
Furthermore, the enhanced methodology will be validated through the application to a different and larger case study, distinct from the virtualized testbed used previously. This real-world scenario will provide a robust validation of the methodology's effectiveness in a practical setting, ensuring its applicability and reliability in diverse industrial environments. 

\section{Outline}
\label{sec:1_outline}
\noindent The thesis is structured as follows:

\begin{description}
	\item [Chapter 2:] provides a background on Industrial Control Systems, (ICSs) describing their structure, components, and some of the network communication protocols used;
	\item [Chapter 3:] following an introductory section that provides a brief overview of the existing literature on process comprehension in industrial control systems, this chapter focuses on a specific paper that outlines a methodology to attaining process comprehension of an industrial system by employing dynamic blackbox analysis;
	\item [Chapter 4:] outlines a proposal to improve and extend the methodology outlined in the previous chapter;
	\item [Chapter 5:] presents the case study on which the proposed methodology will be applied;
	\item [Chapter 6:] shows how the proposed methodology is applied to the case study illustrated above;
	\item [Chapter 7:] outlines final conclusions and future work.
\end{description}

\vfill
\nolinenumbers

%###### CAPITOLO 2 ######%
\chapter{Altro Capitolo}
Ri-bla \cite{einstein}

\chapter{Test include}
\label{test}

Lorem ipsum dolor e non mi ricordo più come va avanti. Che tanto manco è latino...


%###### BIBLIOGRAFIA ######%
\printbibliography[heading=bibintoc, title=Bibliografia]

%% Vecchia versione
%\begin{thebibliography}{99}
%	\addcontentsline{toc}{chapter}{Bibliografia}	
%\end{thebibliography}

%%%%%%%%%%%%%%%%%%%%%%%%%%%%%%%%%
%		Fine TESI				%
%%%%%%%%%%%%%%%%%%%%%%%%%%%%%%%%%
\end{document}